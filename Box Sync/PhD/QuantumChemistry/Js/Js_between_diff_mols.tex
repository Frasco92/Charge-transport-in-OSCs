 \documentclass[a4paper,12pt]{article}
 \usepackage[table,xcdraw]{xcolor}
\usepackage[margin=2cm]{geometry}
\usepackage{amsmath}
\usepackage{amsfonts}
\usepackage{graphicx}   
\usepackage{float}
\usepackage[center]{caption}
\usepackage{subcaption}
\usepackage[square,sort,comma,numbers]{natbib}
\usepackage{tikz,pgfplots}
\usepackage{filecontents}
\usepackage{wrapfig}
\usepackage{upgreek}
\usepackage{appendix}
\usepackage{listings}
\usepackage{color}
\setlength\parindent{0pt}

\pgfplotsset{compat=1.8}
\newcommand{\HRule}{\rule{\linewidth}{0.5mm}}

\begin{document}

\section{Projective Method}

The \textit{projective method} for finding transfer integrals (or electronic couplings) between molecules involves projecting the orbitals of a pair of molecules onto a basis set definied by the orbitals of the individual molecules. The molecular orbital energies of the pair of molecules are then used to rewrite the Fock matrix in the new localised basis set. If the atomic orbitals do not form an orthogonal set, they must first be re-orthogonalised with a method such as Lowdin orthogonalisation. With this method, it is not necessary to assume that the two energies of the HOMOs (or LUMOs) are equal or assume anything about the orbitals used to define the isolated molecules Slater determinant.\\

For a general case of the transfer integral between two different molecules, the procedure is as follows:

\begin{itemize}

\item The molecular orbitals and overlaps of each of the molecules and the pair of molecules (as well as the eigenvalues of the pair of molecules) are calculated with \textit{Gaussian 09} using the command \# p b3lyp/6-31g* punch=mo iop(3/33=1). This uses the B3LYP level of theory and the 6-31g* basis set. The punch keyword indicates that the MOs should be sent to a separate output file and iop(3/33)=1 means that one-electron integrals should be printed. 

\item The orbitals of the of the pair of molecules are written as a matrix $\textbf{M}_{pair}$, where each column corresponds to a molecular orbital and each row to an atomic orbital. The orbitals of the isolated molecules are written as a matrix $\textbf{M}_{loc}$,  which is the basis that the $\textbf{M}_{pair}$ orbitals will be projected onto, and can be written as a block diagonal matrix:

 \[  \textbf{M}_{loc} = \left( \begin{array}{ccccccccc}
\phi_1^1 &  \phi_1^2 & \phi_1^3 & ... & \phi_1^m  & 0 & 0 & 0 & 0\\
\phi_2^1 &  \phi_2^2 & \phi_2^3 & ... & \phi_2^m & 0 & 0 & 0 & 0 \\
\phi_3^1 &  \phi_3^2 & \phi_3^3 & ... & \phi_3^m & 0 & 0 & 0 & 0 \\
... & ... & ... & ... & ...& 0 & 0 & 0 & 0\\
\phi_m^1 &  \phi_m^2 & \phi_m^3 & ... & \phi_m^m & 0 & 0 & 0 & 0\\
0 & 0 & 0 & 0 & 0 & \phi_1^1 &  \phi_1^2 & ... & \phi_1^n\\
0 & 0 & 0 & 0 & 0 & \phi_2^1 &  \phi_2^2 & ... & \phi_2^n\\
0 & 0 & 0 & 0 & 0 & ... & ... & ... & ... \\
0 & 0 & 0 & 0 & 0 & \phi_n^1 &  \phi_n^1 & ... & \phi_n^n\\
 \end{array} \right)\] 


 where $\phi_i^j$ represents the contibution of the $j^{th}$ molecular orbital to the $i^{th}$ atomic orbital, and the first $m$ columns/rows correspond to the first molecule, and the next $n$ columns/rows correspond to the second molecule.  \\

\item If the molecular orbitals are non-orthogonal, they can be orthogonalised with Lowdin orthogonalisation. Defining the matrix $\textbf{D}$ as the result of Cholesky decomposition of the overlap matrix \textbf{S} :

\begin{equation}
\textbf{S}=\textbf{D}^T\textbf{D}
\end{equation}

where the $i^{th}$, $j^{th}$ element $S_{i,j}$ of the overlap matrix correspond to the integral over all space of the product of the $i^{th}$ and $j^{th}$ atomic orbitals. The non-orthogonal orbitals $\textbf{M}_{loc}'$ are orthogonalised with 

\begin{equation}
\textbf{M}_{loc}=\textbf{D}\textbf{M}_{loc}'
\end{equation}

\item Now, to write the orbitals of the pair of molecules in the new basis of the orbitals of the isolated molecules, it is simply the dot product of the localised orbitals and the orbitals of the pair:

\begin{equation}
\textbf{M}_{pair}^{loc}=\textbf{M}_{loc}^{T} \textbf{M}_{pair}
\end{equation}

\item Finally, the Fock matrix for the pair of molecules can be written as:

\begin{equation}
\textbf{F}_{pair}^{loc}=\textbf{M}_{pair}^{loc T} \epsilon_{pair} \textbf{M}_{pair}^{loc}
\end{equation}

where $\epsilon_{pair}$ is a diagonal matrix containing the molecular orbital energies of the pair of molecules. The off-diagonal elements $i$, $j$ of this matrix represent the transfer integral between the $i^{th}$ and $j^{th}$ molecular orbitals of the isolated molecules, so if the HOMO is known to be the $q^{th}$ molecular orbital, the HOMO-HOMO coupling is the $F_{q, q+m}^{loc}$ element.    

\end{itemize}








\end{document}